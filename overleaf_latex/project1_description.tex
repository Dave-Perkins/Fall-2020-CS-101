\documentclass[11pt, letterpaper, onecolumn, oneside, final]{article}


\usepackage{handout}
\usepackage{soul}

\newfontfamily{\consolas}{Consolas}[Extension = .ttf]

\DocumentTitle {Project 1}
\DocumentSubtitle {Audio!}
% End of preamble

%%%%%%%%%%%%%%%%%%%%%%%%%%%%%%%%%%%%%%%%%%%%%%%%%%%%%%%%%%%%%%%%%%%%%%%

\begin{document}
    \maketitle

\section{Assignment Goals.} 

In this assignment, the students will be expected to demonstrate their understanding of the following concepts:
\begin{itemize}
\item \textbf{List access and manipulation techniques}
\begin{itemize}
\item Students should be able to show that they understand the structure of lists and how
to access their data. They should also demonstrate that they understand how to effectively manipulate lists by means of pulling relevant data out through iteration, slicing, and indexing.
\end{itemize}
\item\textbf{Use of audio methods and operators}
\begin{itemize}
\item Student demonstrates knowledge and use of the {\consolas cs101audio} package or other, more
advanced, {\consolas PyDub} functions.

\end{itemize}
\item \textbf{Control flow constructs \& logic}
\begin{itemize}
\item Students should demonstrate an understanding of control flow constructs, at the minimum they should a understanding of for loops that go beyond the basics, but they should also demonstrate an basic understanding of if/elif/else blocks and basic logic, and how these constructs are used to control the flow of a program.
\end{itemize}
\end{itemize}

Through the code they submit, as well as in their grading meeting, they should be able to demonstrate their understanding of the code that they wrote and the aforementioned computer science principles to at least a satisfactory degree.

In order to complete the project, students should be taught the following concepts:
\begin{itemize}
    \item {\consolas cs101audio} methods
    \item List basics (indexing, splicing, append, $\dots$)
    \item Control flow constructs (basic if/complex for)
\end{itemize}

\end{document}
