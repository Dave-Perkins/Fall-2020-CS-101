\documentclass[11pt, letterpaper, onecolumn, oneside, final]{article}

\usepackage{lab}
\usepackage{soul}

\newfontfamily{\consolas}{Consolas}[Extension = .ttf]

\DocumentTitle {Project 5}
\DocumentSubtitle {Web-Scraping!}
% End of preamble

%%%%%%%%%%%%%%%%%%%%%%%%%%%%%%%%%%%%%%%%%%%%%%%%%%%%%%%%%%%%%%%%%%%%%%%

\begin{document}
    \maketitle

    %\duedate{Weekday assigned: 0-6}{Week assigned: 0}{Weekday due:0-6}{Week due: 1}{Time due: 10:00 p.m.}
    
    \duedate{4}{0}{0}{3}{10:00 p.m.}

    \section{Collaboration.} Reminder of the collaboration policy: you may discuss the ideas of the project, but cannot share code or look at another student’s code. If you discuss, cite. More details are in the course syllabus. 

    \section{Introduction.} In this project, you will be exploring the {\consolas beautifulsoup4} module and the process of web-scraping with Python.
     
    \section{What to do.}
    Your task for this project is to create a program that meaningfully scrapes data from a website of your choosing, parses it, and performs some task on the data that you've gotten. Some good sites to scrape include:
\begin{itemize}
    \item IMDb,
    \item Amazon,
    \item SoundCloud, and
    \item Wikipedia.
\end{itemize}
    and outputs this data into a readable format. This output can be directly printed in the shell or written to a file for greater readability.
    \\ 
    As always, you may draw upon your knowledge from lab and lectures, and you may look further into {\consolas beautifulsoup4} methods and applications, but be sure to cite when necessary. As a reminder, here is what a citation should look like:\\
    \\
    \indent{\consolas \# CITE: https://docs.python.org/3/library/turtle.html}\\
    \indent{\consolas \# DETAILS: Looked up how to change the background color.}\\ 
    \\
    Your final product should demonstrate what you have learned in this course over the semester as well as an understanding of the {\consolas beautifulsoup4} module. While you do not need to show every single concept you have learned throughout the semester, you should submit a piece of code that represents your understanding of the overarching themes of computer science.
    Submissions that reflect a deeper understanding or more interesting use of these techniques will be graded more favorably. Some options to display a more advanced grasp on the necessary topics include:
    
\begin{itemize}
    \item Advanced web-scraping techniques
    \item Advanced general coding techniques
\end{itemize}

Refer to the Grading Rubric below for more grading information.
\newpage
\section{Examples.}
\begin{itemize}
    \item Pull prices of similar products from an online marketplace
    \item Pull data from a tables to, analyze it in some capacity, then export results into a CSV
    \item 
\end{itemize}

\section{How to submit.}
Submit your project to Gradescope using the standard course submission procedures. 
\section{Grading Rubric:} 
    \begin{enumerate}
        \item \textbf{Creativity}: Your program should do something innovative or unique using interesting data from at least one source. Simply scraping some data and outputting it is \textbf{not} enough.
        \item \textbf{Meaningful analysis \& presentation of data}: Your program, once it has the data, should strive to do something meaningful with it. It should solve a problem, or automate an interesting task and then display the results. 
        \item \textbf{Use of web scraping techniques}: Your program should incorporate the {\consolas beautifulsoup} module to scrape data from websites. You must scrape data from at least two pages on at least one site. 
        \item \textbf{Use of course content}: As this is the last project of the semester, you should be able to display what you have learned over the course of the semester and that you have a solid understanding of the concepts you have learned. This \textbf{does not} mean your code must display aspects from every single unit but instead your code should be an example of your ability to effective write code as a whole. 
    \end{enumerate}
    \section{Important Notes:} 
    \begin{enumerate}
        \item You must be able to explain your code in a reasonable amount of detail to the grader that you will be meeting with. This will demonstrate to them that you understand the code that you have written. If you are unable to talk through your code, it is probably a sign that you need to visit TA hours or stop into your Professor's office hours for clarification. 
    \end{enumerate}

\end{document}
