% NLTK 2
% GRIDS?
\documentclass[11pt, letterpaper, onecolumn, oneside, final]{article}

\usepackage{lab}
\usepackage{soul}

\newfontfamily{\consolas}{Consolas}
[Extension = .ttf]
%\usepackage{fontspec}

\DocumentTitle {Lab 2B}
\DocumentSubtitle {Grids!}
% End of preamble

%%%%%%%%%%%%%%%%%%%%%%%%%%%%%%%%%%%%%%%%%%%%%%%%%%%%%%%%%%%%%%%%%%%%%%%

\begin{document}
\maketitle
In this lab you will be working with lists of lists, or \textbf{grids}. These are valuable data structures that will come in handy for your next project.
% MAKE SURE TO SAVE THIS LAB IN A LABS FOLDER
% LISTS/FOR LOOP QUESTIONS -------------------------------------------
\begin{enumerate}
\item You will be entering the following commands in the shell window. Begin by typing:

\verb|grd = [['lovelace', 1815], ['turing', 1912], [0, 1]]|.

You may find the following visualization of the grid helpful for your predictions:
\begin{lstlisting}
            grd = [['lovelace', 1815], 
                   ['turing',   1912], 
                   [0,          1   ]].
\end{lstlisting}
\begin{enumerate}
    \item Predict (and then check) the value of \verb|grd[1][1]|.
    \item Try \verb|grd[0][0]|.
    \item Try \verb|grd[1][0][3]|.
    \item Try \verb|grd[0][grd[2][1]]|.
    \item What expression would yield \verb|'love'|?
    \item Try \verb|grd.index(1912)|. 
    \item Try \verb|grd[1].index(1912)|.
    \item Try \verb|grd.index(['turing', 1912])|.
\end{enumerate}

\item Add code to this doubly nested for loop that will fill each spot in the grid with a riding number so that the final list of lists looks like {\consolas [[0,1,2],[3,4,5],[6,7,8]]}.
\begin{center}
 \colorbox{lightgray}{\parbox{.9\textwidth}{\consolas\\
grid = [[0,0,0],[0,0,0],[0,0,0]]\\
count = 0\\
for x in range(len(grid)):\\
\hspace*{6mm}for y in range(len(grid[x])):\\
\hspace*{12mm}\# YOUR CODE HERE}}
\end{center}
\vspace*{10mm}
 \item Write a function {\consolas ticTacWin} that can detect a win on a 3 $\times$ 3 tic-tac-toe board, which is represented as a grid containing `O's, `X's, and `\_'s if the space is empty.\\
\\
\newpage
\item Write the function \verb|mult_table| to create a multiplication table for the numbers \verb|1| through \verb|9|. The output should look something like this:
\begin{lstlisting}
                    1   2   3   4  
                    2   4   6   8  
                    3   6   9  12  
                    4   8  12  16
\end{lstlisting}


\item Write a function {\consolas neighbor\_sum} that takes a grid and returns a new grid where each element is the sum of all of its surrounding neighbors. Note that not all elements have 8 neighbors. (Hint: Check each neighbor's coordinates to make sure that the element is still in the grid. For example, (-1, 2) would not be in the grid, since there is no -1 row.)\\
Example:

Given the grid below:
\begin{lstlisting}
                    1   2   3   4  
                    2   4   6   8  
                    3   6   9  12  
                    4   8  12  16
\end{lstlisting}

The returned grid should be:

\begin{lstlisting}
                    8   16  24  17  
                    16  32  48  34  
                    24  48  72  51  
                    17  34  51  33
\end{lstlisting}






\item Write a function {\consolas replaceTarget} that takes a grid and a target number and then returns a grid in which all instances of that target number have been replaced with a zero.
\begin{center}
 \colorbox{lightgray}{\parbox{.99\textwidth}{\consolas
def replaceTarget(grid, grid):\\
\hspace*{6mm}\# YOUR CODE HERE}}
\end{center}

\item Create a {\verb fill_grid } function that takes dimensions as a tuple {\consolas (row and col)}, and something to fill the grid in with. The function returns a grid with the correct number of rows and columns, filled in with the second parameter. \\
Example: The function call {\consolas fill\_grid((3,4), 'H')} returns the grid:
\begin{lstlisting}
                    H H H H
                    H H H H
                    H H H H.
\end{lstlisting}
\newpage
\item Call the {\verb fill_grid } function and use strings containing periods as the fill. Then write the {\verb one_and_two } function, which should only have one parameter (the grid), and should choose one random location in the grid to be \verb|1| and another to be \verb|2|. Don't forget to \verb|import random| at the top of your program. The final output should resemble something like this:
\begin{lstlisting}
                    . . . . 
                    . 1 . . 
                    2 . . .
\end{lstlisting}

\end{enumerate}
\end{document}