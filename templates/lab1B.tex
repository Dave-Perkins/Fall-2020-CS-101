% AUDIO 2
\documentclass[11pt, letterpaper, onecolumn, oneside, final]{article}

\usepackage{lab}
\usepackage{soul}

\newfontfamily{\consolas}{Consolas}
[Extension = .ttf]
%\usepackage{fontspec}

\DocumentTitle {Lab 1B}
\DocumentSubtitle {Booleans \& Functions!}
% End of preamble

%%%%%%%%%%%%%%%%%%%%%%%%%%%%%%%%%%%%%%%%%%%%%%%%%%%%%%%%%%%%%%%%%%%%%%%

\begin{document}
\maketitle

% MAKE SURE TO SAVE THIS LAB IN A LABS FOLDER
 In this lab you will learn about the uses and applications of Boolean values and the basics of function creation.
\begin{enumerate}
\item Predict the result of each of these expressions and then try them out in the shell window,
assuming {\consolas planets = 8}. Each of these returns a \textbf{Boolean}.
\begin{enumerate}
    \item {\consolas planets == 9}
    \item {\consolas planets < 9}
    \item {\consolas planets <= 8}
    \item {\consolas planets != 7}
    \item {\consolas 8 != planets}
    \item {\consolas (planets >= 5) and (planets < 20000)}
    \item {\consolas (planets != 8) or (planets == 9)}
    \item {\consolas True and (False or (not False))}
\end{enumerate}
\item Try running the following code segment. Try to figure out why it doesn't work and see if you can fix it. \\ \\
    \colorbox{lightgray}{\parbox{.99\textwidth}{\consolas i\_like\_python = True \\ 
    if (i\_like\_python = True): \\
    \hspace*{6mm}print("Python rules!")\\ else: \\
    \hspace*{6mm}print("I think C++ is better...")}}\\\\\\
\item Take a look at the code segment below. Then, based on the given values of the variables {\consolas rain} and {\consolas wednesday}, predict the value of 
{\consolas biking}.\\

\colorbox{lightgray}{\parbox{.88\textwidth}{\consolas if not rain: \\
\hspace*{6mm}if wednesday == False: \\ \hspace*{12mm}biking = True \\ \hspace*{6mm}else:\\ \hspace*{12mm}biking = False \\else: \\ \hspace*{6mm}if wednesday == True:\\ \hspace*{12mm}biking == True \\ \hspace*{6mm}else: \\
\hspace*{12mm}biking == False}}

\begin{enumerate}
    \item {\consolas rain = False}, {\consolas wednesday = False}\\
    \item {\consolas rain = False}, {\consolas wednesday = True}\\
    \item {\consolas rain = True}, {\consolas wednesday = False}\\
    \item {\consolas rain = True}, {\consolas wednesday = True}\\
\end{enumerate}
\\
\textbf{Show a TA or Professor your solutions and get a sticker!}
\item Try running the code segment below.

\colorbox{lightgray}{\parbox{.88\textwidth}{\consolas result = "" \\ x = 17 \\ 
if x > 5: \\ \hspace*{6mm} result += "He" \\ \hspace*{6mm}if x < 15: \\ 
\hspace*{12mm}result += "ad" \\else: \\ \hspace*{6mm}result += "y" \\ 
print(result)}}

What is the output? How can we change the code to make it so when {\consolas x} is greater than or equal to 15, the output is \textquotedblleft Hey"?\\\\

\item 
    What does the code below do? Try running it. (Remember: you can stop a program while it's running by pressing the red \textquotedblleft Stop" button in 
    Thonny.) \\ \\
    \colorbox{lightgray}{\parbox{.88\textwidth}{\consolas x = 10 \\ while(x >= 10): \\ 
    \hspace*{6mm} print("X =", x) \\ \hspace*{6mm} x += 1}} \\ 


    Why does this happen? Try fixing this code segment so that it prints only the numbers between 0 and 10.\\\\\\\\\\

\textbf{Show a TA or Professor your functions and get a sticker!}
%\item Do the each of the following in your shell window.
%\begin{enumerate}
%\item Type {\consolas my\_input = input("Enter a number: ")}.
%\item Type your favorite number and hit enter.
%%\item Convert {\consolas my\_input} to an {\consolas int}.
%\end{enumerate}
%\newpage
% FIND A TEXT FILE FOR THEM TO DOWNLOAD AND PUT IN THE SAME FOLDER AS THE LAB SO THEY CAN PRACTICE OPENING/READING FILES
%\item In your editor, type the following lines of code.
%\begin{center}
%\colorbox{lightgray}{\parbox{.6\textwidth}{\consolas\\
%file = open("filename.txt")\\
%text = file.read()\\
%print("TEXT:", text)\\
%caps = text.upper()\\
%print("CAPS:", caps)\\
%lowers = text.lower()\\
%print("LOWERS:", lowers)}}
%\end{center}
%    \begin{enumerate}
%        \item What is {\consolas text}? What is the type of {\consolas text}?\\\\
%        \item What is the difference between {\consolas text} and {\consolas caps}? What does {\consolas .upper()} do?\\\\
%        \item What is the difference between {\consolas text} and {\consolas lowers}? What does {\consolas .lower()} do?\\\
%    \end{enumerate}
%\end{enumerate}

\newpage
\item For each of the functions below, describe what you believe it does and then run it to see if you are correct.
\begin{enumerate}
\item
\colorbox{lightgray}{\parbox{.6\textwidth}{\consolas
def mystery1(vars1):\\
\hspace*{8mm}myVars = []\\
\hspace*{8mm}for x in var1:\\
\hspace*{16mm}if(x \% 2 == 0):\\
\hspace*{24mm}myVars.append(x)\\     
\hspace*{8mm}return myVars
}}\\\\\\
\item
\colorbox{lightgray}{\parbox{.6\textwidth}{\consolas
def mystery2(var2):\\
\hspace*{8mm}for x in range var2:\\
\hspace*{16mm}print("*"*x)
}}\\\\\\
\item What output will {\consolas mystery3(2,[10,30,50,24,88,99,104,23,9,2])} generate?\\\\
\colorbox{lightgray}{\parbox{.6\textwidth}{\consolas
def mystery3(var1, var2):\\
\hspace*{8mm}return mystery4(var2,var1)\\

def mystery4(var1, var2):\\
\hspace*{8mm}myVars = []\\
\hspace*{8mm}for x in range(0,len(var1),var2):\\
\hspace*{16mm}myVars.append(var1[x])\\
\hspace*{8mm} return myVars
}}\\\\\\
\end{enumerate}
    %item In Python you can create blocks of code called functions. Functions allow you to write code that can be run from anywhere in your code without having to type it all out again. To write a function, Python has a specific syntax that you must use which looks like this:
%\begin{center}
%\colorbox{lightgray}{\parbox{.25\textwidth}{\consolas
%    def sum(num1, num2):\\
%    \hspace*{8mm}sum = num1 + num2\\
%    \hspace*{8mm}return sum}}.\\
%\end{center}
%    The above chunk of code defines a function called {\consolas sum} which takes two numbers and returns their sum. You can call this function and store the value it returns like this:
%\begin{center}
%\colorbox{lightgray}{\parbox{.225\textwidth}{\consolas
%   mySum = sum(10,20)}}.
%\end{center}
%    To call a function you have defined, type the function name followed by parentheses, which should contain the arguments you want to give the function. This function takes two parameters: {\consolas num1} and {\consolas num2}. In this function call, 10 is being passed as {\consolas num1} and 20 as {\consolas num2}. This function then adds those two numbers and returns the sum which gets stored in {\consolas mySum}. 
    
    
\item You will now be writing some of your very own functions!

\begin{enumerate}
\item Write the function {\consolas concatenate} that takes a list of strings and returns all of the strings in the list concatenated together.\\ Example:\\
{\consolas >>> concatenate(["Hi", "I'm", "a", "happy", "penguin"])\\ "HiI'mahappypenguin"}\\

\item Write the function {\consolas my\_max} that takes 2 integers as arguments, and returns the larger of the two. You may not use the built-in max function.\\ Example: \\
\hspace*{6mm}{\consolas >>> my\_max(15, 7) \\
\hspace*{6mm}15}\\


\item Write the function {\consolas max\_list} that takes a list of integers and returns the largest integer in the
list. You may (but don’t have to) use your {\consolas my\_max} function; do not use the built-in max function. \\
Example: \\ 
\hspace*{6mm}{\consolas >>> max\_list([43, 111, 6, 341, 5, 31, 143, 5, 143]) \\
\hspace*{6mm}341}\\

\item Unlike the functions you've just written, not all functions return a value. A great example of this is the classic first function many programmers learn, writing a function that prints \textquotedblleft Hello, World". Write a function that prints out \textquotedblleft  Hello, World" but returns nothing. Once you've done this, change the function to take one parameter that effects the printed string in some way. For example, you could pass a number that is the number of times \textquotedblleft Hello, World" is printed. You could also pass a string which gets printed in the place of \textquotedblleft World".\\

\item What is the sum of the first 10 perfect squares?\\\\
Use this answer as a test case for a function {\consolas sum\_squares} that asks for a number {\consolas num\_squares} and returns the sum of the first {\consolas num\_squares} perfect squares. Test your function by calling it a few times in the main function!\\

\item Write a function that asks for a string {\consolas my\_string} and a single symbol {\consolas my\_symbol}. Your program should replace every second letter in {\consolas my\_string} with {\consolas my\_symbol}, and return that new string.
For example, if the user inputs {\consolas WIPPMAN} and {\consolas /}, then the output should be {\consolas W/P/M/N}.\\

\item Write a function that takes a string and returns the reverse of that string.\\

\item Write a function that checks to see if a string is a palindrome. As a reminder, a palindrome is a word that is the same forward and backwards. (Hint: use your string reverse function you just wrote!)\\\\
\end{enumerate} 
\textbf{Show a TA or Professor your functions and get a sticker!}
\end{enumerate}

\end{document}