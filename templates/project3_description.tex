\documentclass[11pt, letterpaper, onecolumn, oneside, final]{article}


\usepackage{handout}
\usepackage{soul}

\newfontfamily{\consolas}{Consolas}[Extension = .ttf]

\DocumentTitle {Project 3}
\DocumentSubtitle {DNA Matching}
% End of preamble

%%%%%%%%%%%%%%%%%%%%%%%%%%%%%%%%%%%%%%%%%%%%%%%%%%%%%%%%%%%%%%%%%%%%%%%

\begin{document}
    \maketitle
\section{Assignment Goals.}
In this assignment, the students will be expected to demonstrate their understanding of the following concepts:
\begin{itemize}
\item \textbf{Parsing \& working with CSVs}
\begin{itemize}
    \item Students will use the {\consolas csv} module to parse the database of DNA that we give them. They will be exposed to and work with large sets of data that might be overwhelming if being parsed by any other method.
\end{itemize}
\item \textbf{Advanced problem solving}
\begin{itemize}
    \item Students must implement their own algorithm to solve this problem. Their algorithm must work thoroughly and they must test their code in more than just the simplest of environments. 
\end{itemize}

\item \textbf{Writing code to perform a specific task}
\begin{itemize}
    \item Students will be required to write code that correctly matches DNA samples. This project is not open-ended, separating it from the other projects thus far in the course. Students must accomplish this task, and the bulk of their grade depends on the ability of their code to generate correct answers. The structure of this project is more in line with the structure of the rest of the department courses.
\end{itemize}

\item \textbf{Benchmark to determine if students should continue in Computer Science}
\begin{itemize}
    \item This project is more along the lines of a standard Hamilton computer science assignment. It is due just before registration and gives students a look into what other courses in the department would be like. Used as an important factor in determining legitimate computer science interest.
\end{itemize}
\end{itemize}
Through the code they submit, as well as in their grading meeting, they should be able to demonstrate their understanding of the code that they wrote and the aforementioned computer science principles to at least a satisfactory degree.\\
\\
In order to complete the project, students should be taught the following concepts:
\begin{itemize}
    \item More complex control flow constructs(while loops, nested if, elif, else etc.) and Boolean logic
    \item {\consolas nltk} \& {\consolas matplotlib} modules
    \item Input/Output Basics
    \item Basics of function definition and use
    \item String manipulation and methods
\end{itemize}

\end{document}
