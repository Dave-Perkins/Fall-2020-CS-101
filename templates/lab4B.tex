%???????????????????
\documentclass[11pt, letterpaper, onecolumn, oneside, final]{article}

\usepackage{lab}
\usepackage{soul}

\newfontfamily{\consolas}{Consolas}
[Extension = .ttf]
%\usepackage{fontspec}

\DocumentTitle {Lab 4B}
\DocumentSubtitle {Objects \& Documentation!}
% End of preamble

%%%%%%%%%%%%%%%%%%%%%%%%%%%%%%%%%%%%%%%%%%%%%%%%%%%%%%%%%%%%%%%%%%%%%%%

\begin{document}
\maketitle

% MAKE SURE TO SAVE THIS LAB IN A LABS FOLDER
% LISTS/FOR LOOP QUESTIONS -------------------------------------------
\section{Introduction.} \\
In this lab you will be further introduced to the concept of \textbf{objects}. Objects are things in Python that  you have actually been using this entire semester. To put it simply, an object is a collection of data and methods (Think of turtles!). Today you will be working with a new object that uses Turtles to learn more about objects. You will also be investigating the {\consolas datetime} and {\consolas calendar} modules to get you used to working with documentation. The documentation for the {\consolas datetime} module can be found at 
\begin{center}
    \newline\textcolor{blue}{\underline{https://docs.python.org/3/library/datetime.html\#module-datetime}}
\end{center}
and the documentation for the {\consolas calendar} can be found at
\begin{center}
    
\newline\textcolor{blue}{\underline{https://docs.python.org/3/library/calendar.html}}.
\end{center} You may use more than these two resources, but the information on these pages should be enough for you to complete the following exercises.\\
To help you get better grasp on how objects work and get you used to working with documentation, there will be minimal explanation in this lab. You will be expected to figure out anything you don't understand by going through the documentation and finding answers to your problems and questions. Try to rely less on TAs for this lab as the point of this lab is to find answers on your own. That being said, if you are really stuck on a problem, the TAs are still here to assist you.

\section{Exercises.} Using the resources above, complete each of the following exercises.
\section{Part 1.}
\begin{enumerate}

    \item Look at the documentation and the code for the {\consolas Turtle\_Drawing} class. What do you think this class does (generally)?
    
    \item Using the {\consolas Turtle\_Drawing} class, draw a triangular Spirograph, with a pen size of 10,  a speed of 5, and a triangle size of 200.
    
    \item Then, using the same {\consolas Turtle\_Drawing} object (without changing the original object creation), draw a square Spirograph with a pen size of 10, a square length of 300, and a doubled speed.
    
    \item Take a look at the {\consolas draw\_recursion} method. What do you think this method does and how? Try and make an instance of the {\consolas Turtle\_Drawings} class which can generate a recursive pentagon drawing.
    
    \item Look through the class provided and try to generate a neat design which you have not already created.
    
    \item Add a method to the {\consolas Turtle\_Drawings} class that when called creates a drawing of you own design.
\end{enumerate}
    
    
\section{Part 2.}
\begin{enumerate}
    \item Given a date at any point in time, write a function to return what day of the week it is.
    
    \item Using the {\consolas calendar} and {\consolas datetime} modules, write a function that returns a list of date objects for every Friday the $13^{th}$ in a given year.
    
    \item Given a date in the future, write a function that returns the number of days until that date.
    
    \item Given two dates, write a function that calculates and returns the number of days between these two dates.
    
    \item Given a date in the past, write a function to calculate how many days there have been since that date until the current day.
    
    \item Take one of the three functions you just wrote and modify what it returns so that it does not just return a total number of days, but formats the output to give it in years, months, and days.
    

\end{enumerate}

\end{document}