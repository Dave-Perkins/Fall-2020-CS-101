% NLTK 1
\documentclass[11pt, letterpaper, onecolumn, oneside, final]{article}

\usepackage{lab}
\usepackage{soul}

\newfontfamily{\consolas}{Consolas}
[Extension = .ttf]
%\usepackage{fontspec}

\DocumentTitle {Lab 2A}
\DocumentSubtitle {Plots \& Words!}
% End of preamble

%%%%%%%%%%%%%%%%%%%%%%%%%%%%%%%%%%%%%%%%%%%%%%%%%%%%%%%%%%%%%%%%%%%%%%%

\begin{document}
\maketitle

% MAKE SURE TO SAVE THIS LAB IN A LABS FOLDER
 In this lab you will learn about the Python {\consolas nltk} and {\consolas matplotlib} modules.\\
 
\section{Installation:}
    First install both {\consolas matplotlib} and {\consolas nltk} from Thonny's package manager, following these steps:
\begin{enumerate}
    \item Open Thonny. Click {\consolas Tools > Manage Packages...}
    \item In the search bar, type {\consolas nltk}. Click {\consolas Install} and wait for it to be installed. You will be able to tell when it is done.
    \item In the search bar, type {\consolas matplotlib}. Click {\consolas Install} and wait for it to be installed. You will be able to tell when it is done.
\end{enumerate}
This is the general process to install most packages in Thonny. The {\consolas nltk} module requires a few more steps. Follow the steps below, depending on the operating system you are using. You can also find these steps at
\textcolor{blue}{\underline{https://www.nltk.org/data.html}}. To finish installing {\consolas nltk}:
\begin{enumerate}
\item In your shell window, type {\consolas import nltk}.
\item In your shell window, type {\consolas nltk.download()}. This should open a new window and may take a few seconds to appear.
\item In this new window, in the box next to {\consolas 'Download Directory:'} type whichever of the following corresponds to your operating system:
\begin{itemize}
    \item MacOS: {\consolas /usr/local/share/nltk\_data}
    \item Windows: {\consolas C:\textbackslash nltk\_data}
    \item UNIX: {\consolas /usr/share/nltk\_data}
\end{itemize}
\item Make sure {\consolas all} is selected in the packages field.
\item Wait until all of the lines in the table in the new window have turned green and the box in the bottom right is entirely gray. Close the new window. 
\item Whenever you want to use {\consolas nltk}, put the following lines of code at the very top of your file:
\begin{center}
\colorbox{lightgray}{\parbox{.32\textwidth}{\consolas from nltk import*\\
from nltk.corpus import*}}
\end{center}
\newpage
\item 
Whenever you want to use {\consolas matplotlib}, put the following line of code at the very top of your file:
\begin{center}
\colorbox{lightgray}{\parbox{.45\textwidth}{\consolas from matplotlib import pyplot as plt}}
\end{center}
\end{enumerate}
\section{What to do.}
\begin{enumerate}
\item Download the {\consolas lab2A.py} skeleton and the corresponding {\consolas csv} file from the Resources page on your class Piazza page. 
\item Your first task is to write some functions that will help you parse the data you will be plotting during this lab. These functions will also prove to be useful in your next project. These might be difficult to write, so feel free to ask a TA or professor if you need help or guidance.
\begin{enumerate}
\item Write a function {\consolas remove\_duplicates} that takes a list of either strings or integers and returns a list which only has one instance of each response. (Hint: you want to make a new list and add each thing only once.)
\item Write a function {\consolas get\_counts} that takes a list of either strings or integers and returns a list with the number of occurrences of each unique element. This list of counts should be in the \textbf{same order} as the list that {\consolas remove\_duplicates} returns. (Hint: call your {\consolas remove\_duplicates} function in your {\consolas get\_counts} to get a list of unique elements that you need the counts for.)\\\\
\end{enumerate}
\textbf{Check your answers with us and get a sticker!}\\\\
These are some of the {\consolas matplotlib} functions that you will need the most moving forwards in this lab and in your next project. If you would like more information, you can find the full documentation at
\textcolor{blue}{\underline{https://matplotlib.org/index.html}}. Note that any parameters with an equal sign are \textbf{optional}. If you want to specify these parameters, you must keep the parameter name and equal sign to the left of the value you would like to use.
\begin{enumerate}
\item \colorbox{lightgray}{\consolas plt.bar(labels, values, colors=colorList)}
\begin{itemize}
\item Plots a bar graph, with \textit{labels} as the bars with respective heights of \textit{values}. \textit{Labels} and \textit{values} are lists with entries for each bar you want to create. If you want to specify colors for the bars, you pass a \textit{colorList}.
\end{itemize}
\item \colorbox{lightgray}{\consolas plt.pie(values, labels, colors=colorList, autopct='\%1.1f\%\%')}
    \begin{itemize}
        \item Plots a pie chart with \textit{values} as the counts, \textit{labels} as each slice of the pie. \textit{Colors} can be used to specify the color of each slice. Leave \textit{autopct} as it is if you want the percentages of each slice displayed on the chart.
    \end{itemize}
\item \colorbox{lightgray}{\consolas plt.show()}
    \begin{itemize}
        \item Shows the plot you generated using the above functions.
    \end{itemize}
\item \colorbox{lightgray}{\consolas plt.xlabel(label)}: 
    \begin{itemize}
        \item Sets the label for the x-axis to \textit{label}.
    \end{itemize}
\item \colorbox{lightgray}{\consolas plt.ylabel(label)}:
    \begin{itemize}
        \item Sets the label for the y-axis to \textit{label}.
    \end{itemize}
\item \colorbox{lightgray}{\consolas plt.title(title)}: 
    \begin{itemize}
        \item Sets the title for the chart to \textit{title}.
    \end{itemize}
\item \colorbox{lightgray}{\consolas plt.legend()}: 
    \begin{itemize}
        \item Creates a legend based on \textit{labels} given to previously defined plots.
    \end{itemize}
\end{enumerate}
\newpage
\item Before creating any graphs and working with our data, we need to figure out what format it is in. Print out the variable named {\consolas data}. Investigate this variable. What is the type of {\consolas data}? What does it contain? How can you access the data for each question individually? \\\\\\\\
\textbf{Discuss your answers with us and get a sticker!}
\item Using the above functions in {\consolas matplotlib} and the functions you wrote in Part 1, create a \textbf{bar graph} of the results from the survey for the question `Which dorm do you live in on campus?'. 
\item Using the above functions in {\consolas matplotlib} and the functions you wrote in Part 1, create a \textbf{pie chart} of the results from the survey for the question `What class year are you in?'.\\\\
\textbf{Show us your graphs and get a sticker!}
\item Unfortunately when representing a larger number of categories on a graph with {\consolas matplotlib}, the colors of the different elements get repeated. To avoid this repetition and have each element have its own unique color we can pass in a list of RGB (red, green, blue) values to our plot so that each element is its own color. This can be accomplished by generating a list that is the same length as the number of elements you have. Each element in this list is a three tuple with each number begin a {\consolas float} ranging from 0-1 with each value representing the intensity of either Red, Green, or Blue. The order of the tuple is (R, G, B) which each letter representing the intensity of the three colors. Below is a loop that will generate a list of RGB three-tuples that is the the same length as the list \textit{labels}, meaning that there is one color per label. 

\begin{center}
\colorbox{lightgray}{\parbox{.9\textwidth}{\consolas colors = []\\
r = 0\\
g = .5\\
b = 1\\
for i in range(len(labels)):\\
\hspace*{8mm}colors.append((r,g,b))\\
\hspace*{8mm}r += .02\\
\hspace*{8mm}g += .02\\
\hspace*{8mm}b -= .02}}
\end{center}
Type this code into your file and pass this list into one of your previously generated plots by using the optional color parameter. (Hint: look at the documentation provided above.) After you have gotten this to work for one of you plots, try playing with the code provided to vary the colors generated by the loop.\\
\item Now that you know the basics of plotting with {\consolas matplotlib}, try to plot the other sets of data you have been given and experiment with parameters you had not used previously.\\\\\\
\textbf{Show us what cool graphs you've made and get a sticker!}
\newpage
One interesting way to gather data to plot using {\consolas matplotlib} is by analyzing text. We will be using the {\consolas nltk} module to analyze pieces of text and literature, both in this lab and in your upcoming project. 
\item To familiarize yourself with the {\consolas nltk} module, type each of the following into your shell window and record the output. First, copy and paste each of the following lines:\\
{\consolas import nltk\\
from matplotlib import pyplot as plt\\
text = "The goal of Hamilton’s Computer Science Department is to prepare students to adapt and excel in an ever-changing field by combining a strong foundation in mathematics, logic and language with exposure to the latest innovations in technology.
All computer systems and peripherals controlled by the Computer Science Department are subject to monitoring at all times. Use of any of the Hamilton College CS systems constitutes consent to monitoring. All use must meet the standards set forth in the Hamilton College Code of Student Conduct."}
\\
\\
Now type each of the following into your shell window, and record the output.
\begin{enumerate}
    \item {\consolas words = nltk.word\_tokenize(text)\\words}
    \item {\consolas sents = nltk.sent\_tokenize(text)\\sents}
    \item {\consolas parts\_of\_speech = nltk.pos\_tag(text)\\parts\_of\_speech}
    \item {\consolas fdist1 = nltk.FreqDist(text)\\fdist1}
    \item {\consolas fdist2 = nltk.FreqDist(words)\\fdist2}
\end{enumerate}
\item Based on the output of each of the above lines of code, answer the following questions.
\begin{enumerate}
\item What do you think {\consolas word\_tokenize} does?\\\\
\item What do you think {\consolas sent\_tokenize} does?\\\\
\item What do you think {\consolas pos\_tag} does?\\\\
\item What do you think {\consolas FreqDist} does?\\\\
\item How is {\consolas fdist2} different from {\consolas fdist1}?\\
\end{enumerate}
You can find documentation on further and more complicated {\consolas nltk} functions and applications at \textcolor{blue}{\underline{http://www.nltk.org/book/}} or at \textcolor{blue}{\underline{http://www.nltk.org}}.
\newpage
\item In your shell window, type the following command:
{\consolas fdist1.plot()}. This brings up a graph of a frequency distribution using {\consolas matplotlib}! This is a cool idea, but you now have the ability to create much better looking graphs that {\consolas nltk} does. Put the code that you typed in your shell into the editor and try to create a nicer looking graph for a frequency distribution, using any type of graph you would like. This should give you some good ideas heading into the next project. Start brainstorming some here!\\\\\\\\
\end{enumerate}
\textbf{Show us what you've come up with and get a sticker!}
\end{document}
