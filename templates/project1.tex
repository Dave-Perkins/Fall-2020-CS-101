\documentclass[11pt, letterpaper, onecolumn, oneside, final]{article}

\usepackage{lab}
\usepackage{soul}

\newfontfamily{\consolas}{Consolas}[Extension = .ttf]

\DocumentTitle {Project 1}
\DocumentSubtitle {Audio}
% End of preamble

%%%%%%%%%%%%%%%%%%%%%%%%%%%%%%%%%%%%%%%%%%%%%%%%%%%%%%%%%%%%%%%%%%%%%%%

\begin{document}
    \maketitle

    %\duedate{Weekday assigned: 0-6}{Week assigned: 0}{Weekday due:0-6}{Week due: 1}{Time due: 10:00 p.m.}
    \duedate{4}{0}{0}{3}{10:00 p.m.}

    \section{Collaboration.} Reminder of the collaboration policy: you may discuss the ideas of the project, but cannot share code or look at another student’s code. If you discuss, cite. More details are in the course syllabus. 

    \section{Introduction.} In this project, you will be exploring the {\consolas cs101audio} module (built off of Python's {\consolas PyDub} module) in order to create and modify audio in Python. 
     
    \section{What to do.} Begin by downloading the necessary skeleton code ({\consolas audio.py} and {\consolas cs101audio.py}) found on the resources page on the class Piazza page. You can find them on the Resources page under Projects. Place them in a Project1 folder in the Projects folder in your CS101 directory. \textbf{Make sure {\consolas audio.py} and {\consolas cs101audio.py} are in the same folder or your code will not work.} You can then open {\consolas audio.py} in Thonny and begin the project. Be sure to save regularly, in case of any mishaps. Make sure to pay attention to the comments already in the file.\\
    \\
    Your task for this project is to create and manipulate sound using the {\consolas cs101audio} package in some interesting way. As always, you may draw upon your knowledge from lab and lectures, and you may look further into {\consolas PyDub} methods and applications, but be sure to cite when necessary. As a reminder, here is what a citation should look like:
    \\ 
    \\
    \indent{\consolas \# CITE: https://docs.python.org/3/library/turtle.html}\\
    \indent{\consolas \# DETAILS: Looked up how to change the background color.}\\ 
    \\
    Your final product should demonstrate your understanding of loops, lists, invoking methods, basic control flow of a program, and the use of {\consolas cs101audio} or {\consolas PyDub} methods and operators. 
    Submissions that reflect a deeper understanding or more interesting use of these techniques will be graded more favorably. Some options to display a more advanced grasp on the necessary topics include:
    
\begin{itemize}
    \item Advanced use of list access/manipulation techniques
    \item Advanced/Interesting use of Audio methods
    \item Advanced use of control flow constructs and logic (loops \& if statements).
\end{itemize}

Refer to the Grading Rubric below for more grading information.
\newpage
\section{Examples.}
% strip silence
% hide a message
% make a song using wave generator
    \section{How to submit.}

    Submit your project to Gradescope using the standard course submission procedures. 
    \section{Grading Rubric:} 
    \begin{enumerate}
        \item \textbf{Creativity}:
        \item \textbf{Diverse use of audio methods}:
        \item \textbf{Use of lists}:
        \item \textbf{Use of loops and ifs}: 

    \end{enumerate}
    \section{Important Notes:} 
    \begin{enumerate}
        \item Note that the effective use of computer science concepts is worth more than the overall aesthetic of the final output. No matter how complicated the audio you create is, it must reflect your coding abilities and grasp of key course concepts. We want to see you experiment and be creative with both your final audio \emph{and} the code that you use to produce it. The amount of effort put into this project will be clear, and the project will be graded accordingly.
        \item You must be able to explain your code in a reasonable amount of detail to the grader that you will be meeting with. This will demonstrate to them that you understand the code that you have written. If you are unable to talk through your code, it is probably a sign that you need to visit TA hours or stop into your Professor's office hours for clarification. 
    \end{enumerate}

\end{document}
