%??????????????
% FUN/EASY REVIEW LAB 

% NUMBER GUESSING GAME (RANDOMLY GENERATE A NUMBER IN A CERTAIN RANGE, HAVE THE USER GUESS UNTIL THEY GET IT RIGHT, EACH TIME NARROWING IT DOWN IF TOO HIGH/TOO LOW)

% INTERACTIVE HANGMAN (PICK A RANDOM WORD FROM SOME LIST WE GIVE YOU & GIVE THE USER A CERTAIN NUMBER OF "LIVES" TO GUESS THE WORD, KEEPING TRACK OF HOW MANY GUESSES & IF THEY ARE IN THE WORD, SHOW THAT SOME HOW)

\documentclass[11pt, letterpaper, onecolumn, oneside, final]{article}

\usepackage{lab}
\usepackage{soul}

\newfontfamily{\consolas}{Consolas}
[Extension = .ttf]
%\usepackage{fontspec}

\DocumentTitle {Lab 2C}
\DocumentSubtitle {More Functions!}
% End of preamble

%%%%%%%%%%%%%%%%%%%%%%%%%%%%%%%%%%%%%%%%%%%%%%%%%%%%%%%%%%%%%%%%%%%%%%%

\begin{document}
\maketitle
In this lab you will be writing a variety of your own functions to perform a wide range of interesting tasks.
% MAKE SURE TO SAVE THIS LAB IN A LABS FOLDER
% LISTS/FOR LOOP QUESTIONS -------------------------------------------
% labs 2 - 5 have decent stuff to take
\begin{enumerate}
\item Write a function that asks for an integer {\consolas last\_number} and, for each integer between 1 and {\consolas last\_number},
prints \textquotedblleft fizz" if the number is divisible by 3, prints \textquotedblleft buzz" if the number is divisible by 5, prints
\textquotedblleft fizzbuzz" if the number is divisible by 3 and 5, and prints just the number otherwise. For example,
if {\consolas last\_number} = 6, then your function should print:
\begin{lstlisting}
1
2
fizz
4
buzz
fizz
\end{lstlisting}

%\item Write a function that asks for a string {\consolas my\_string} and a single symbol {\consolas my\_symbol}. Your program should replace every second letter in {\consolas my\_string} with {\consolas my\_symbol}, and return that new string.
%For example, if the user inputs {\consolas WIPPMAN} and {\consolas /}, then the output should be {\consolas W/P/M/N}.

%\item What is the sum of the first 10 perfect squares?\\
%\\
%You can use that answer as a test case for a function (that you should now write) that asks for a number {\consolas num\_squares} and returns the sum of the first {\consolas num\_squares} perfect squares. Test your function by calling it a few times in the main function!

\item 
Write a function that takes in a sentence and translates it to Pig Latin. If a word starts with a vowel (a, e, i, o, or u), it should have \textquotedblleft ay" added to its end; otherwise, the first letter should be moved to the end of the word, followed by \textquotedblleft ay". This rule should be
applied to every word in the sentence.
Here is a sample interaction with the program:\\
{\consolas Enter a sentence: the swine dined on apples and bananas\\
Pig Latin: hetay winesay inedday onay applesay anday ananasbay}.


%\item Write a function that takes a string and returns the reverse of the word.

%\item Write a function that checks to see if a string is a palindrome. As a reminder, a palindrome is a word that is the same forward and backwards. (Hint: use your string reverse function you just wrote)

\item Write a tic-tac-toe program. For this you will have the inputs be determined by coordinate tuples (where the first number denotes the row and the second denotes the column) that are given using {\consolas input()}. Have the board be printed out every time a move is made and print win when somebody wins. (Hint: Store the \textquotedblleft X"s and \textquotedblleft O"s in a 2D list and don't forget you already wrote a function to tell when a tic-tac-toe game is won--use this!)


\item %maybe rock paper scissors
Write a function that allows the user to play a game of \textquotedblleft Rock, Paper, Scissors" against the computer. The computer should pick a random object  (rock, paper, or scissors). You should take user input to determine which they choose, and then compare it with the random choice of the computer to determine a winner. 

\item Create a function that implements a number guessing game that works as follows:
\begin{itemize}
\item Generate a random number in a range of your choice. Do not output this and do not let the user see it.
\item Allow the user to input numbers also in this range, until the number is guessed.
\item If the number they entered is incorrect, inform them that it is wrong and give some indication as to if the guess was too high or too low.
\item Once the user guesses the correct number, output the number of guesses that it took for them to finish the game.
\end{itemize}

\item The alphabet game is a game played on long car trips, in which the occupants of the car try to find every letter of the alphabet on signs outside of the car. As an adaptation of this game, your task is to write a program that, given a string of text from the user, tells which letters in the alphabet do not appear in the string. Your program should consider both uppercase and lowercase letters to qualify as a letter in the string; in other words, both \textquotedblleft A” and \textquotedblleft a” should be recognized as the letter A in the string. Your program should print, in alphabetical order, the uppercase version of the letters that do not appear in the input.
Here is a sample interaction with the program. Underlined values indicate user input.\\
{\consolas Enter some text: \underline{The quick brown fox ate 4 rotten eggs and said "YUCK!"}\\ Letters not in the text: JLMPVZ}

\item Write a function that creates an interactive game of hangman. Pick a random word and a certain amount of 
\textquotedblleft lives" (or guesses) that the user has to get the word correctly. Output underscores in a line to represent the word, replacing the underscores with letters as those letters are guessed correctly. Continually ask the user for letters, outputting a message and removing a life if it is not in the word, or adding it to the word if it is a correct guess. The game should continue until there are no lives left or until the word is complete.

% INTERACTIVE HANGMAN (PICK A RANDOM WORD FROM SOME LIST WE GIVE YOU & GIVE THE USER A CERTAIN NUMBER OF "LIVES" TO GUESS THE WORD, KEEPING TRACK OF HOW MANY GUESSES & IF THEY ARE IN THE WORD, SHOW THAT SOME HOW)
\end{enumerate}
\end{document}